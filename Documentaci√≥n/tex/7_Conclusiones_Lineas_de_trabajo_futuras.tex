\capitulo{7}{Conclusiones y Líneas de trabajo futuras}

\section{Conclusiones}
Una vez se ha finalizado el proyecto puedo sacar mis conclusiones, y ver tanto la evolución del mismo como las herramientas empleadas para programarlo.

El proyecto ha sido controlado mediante la metodología ágil de SCRUM a través de GitHub, lo que me ha permitido tener un mayor control sobre el proyecto, permitiéndome conseguir un mejor resultado en el mismo tiempo. Esto se debe a que al realizar el proyecto en sprints puedo ir viendo la evolución y corrigiendo los pequeños fallos que iba encontrando en el siguiente sprint.

Al tener que utilizar una encuesta desconocida para mí implica que al principio del proyecto me llevó un tiempo el estudio de su contenido, el pensar y decidir que utilizar y de que manera ya que se disponía de bastante información.

El trabajo ha supuesto un reto para mí, ya que además de utilizar y reforzar los diferentes conocimientos aprendidos durante el grado, he tenido que utilizar nuevas herramientas como pueden ser Heroku o Bokeh, las cuales supusieron un tiempo de comprensión y aprendizaje para así poder utilizarlas. Gracias a estas herramientas ha sido posible crear una aplicación web la cual se ejecuta desde un servidor en internet y es accesible a todas las personas.


\section{Líneas de trabajo futuras}
Algunas de las futuras líneas de trabajo que se podrían realizar serían:

\begin{itemize}
    \item Añadir la posibilidad de cambiar el idioma de la aplicación.
    \item Incluir una opción para que no solo se pueda utilizar los datos de la encuesta del año en el que se ha programado, sino que el usuario pueda escoger entre todos los resultados de encuestas que haya hasta ese momento.
    \item Crear una opción que permita cambiar la aplicación a modo oscuro o a colores mejor adaptados para personas que sufren de daltonismo.
    \item Además del umbral, permitir más tipos de filtros para el sistema de recomendación como elegir el rango de experiencia de los encuestados, filtrar por países, por el tipo de trabajo que desempeña cada uno, entre otros.
    \item Página que tenga todas las tecnologías unidas en una sola red.
    \item Explotar la información disponible en la base de datos que no ha sido empleada para la realización de este TFG, como por ejemplo el trabajo y el perfil del desarrollador.
\end{itemize}